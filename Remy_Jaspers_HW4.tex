\documentclass[10pt,a4paper]{article}
\usepackage[latin1]{inputenc}
\usepackage{amsmath}
\usepackage{amsfonts}
\usepackage{amssymb}
\usepackage{graphicx}
\usepackage[default,osfigures,scale=0.95]{opensans}
\usepackage[T1]{fontenc}
\author{Remy Jaspers - 4499336}
\title{CAO HW4}
\begin{document}
	\maketitle
	\section*{1a}
	The lower 16 bits are put into the sign extender, which in this instruction is: \\
	0000 1000 0010 0101. \\
	The sign bit is copied to fill the remaining 16 bits, thus the output of the sign extend unit will be:\\
	0000 0000 0000 0000 0000 1000 0010 0101. \\
	At the input of the Adder, this value will be shifted left two bits, thus the value will become:\\
	0000 0000 0000 0000 0010 0000 1001 0100\\\\
	The shift left 2 unit will take the lower 26 bits of the instruction and shift these left two bits. The input for this unit is:\\
	1 1000 0010 0000 1000 0010 0101 \\
	Shifted left twice to produce:\\
	1 1000 0010 0000 1000 0010 0101 00\\\\
	This will be the output of the shift left 2 unit. The Jump address will be calculated from concatenating PC+4 to this address. The current PC value is not given so we cannot calculate the upper four bits.\\\\

	\section*{1b}
	\begin{table}[h]
		\centering
		\label{my-label}
		\begin{tabular}{|l|l|}
			\hline
		RegDst	&  1\\ \hline
		ALUSRc	& 0 \\ \hline
		MemtoReg & 0  \\ \hline
		RegWrite&  1 \\ \hline
		MemRead 	&  0\\ \hline
		MemWrite	&  0 \\ \hline
		Branch 	&  0\\  \hline
	 	ALUOp1	&  1\\ \hline
  		ALUOp2	&  0\\  \hline \end{tabular}
	\end{table}
	\section*{1c}
	As this is an R-format instruction, the new PC value will just be PC+4.
	\section*{1d}
	The instruction fetched from the instruction memory is: \\\\
	\verb|or $at, $t4, $v0| \\\\
	The values in these registers are provided. \$t4 = R12 = 16 and \$v0 = R2 = -128. These are the inputs to the main ALU.\\
	16 | -128 = -112 = 0xFFFFFF90 = 1111 1111 1111 1111 1111 111 1001 0000.\\\\
	The inputs for the Adder are (as given in 1a) 0x00002094, from the shift left 2 unit. The other input comes from the PC+4. Thus the adder performs the calculation 0x00002094 + (PC + 4).
	\section*{2a}
\end{document} 