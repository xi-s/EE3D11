\documentclass[10pt,a4paper]{article}
\usepackage[latin1]{inputenc}
\usepackage{amsmath}
\usepackage{amsfonts}
\usepackage{amssymb}
\usepackage{graphicx}
\usepackage[default,osfigures,scale=0.95]{opensans}
\usepackage[T1]{fontenc}
\author{Remy Jaspers - 4499336}
\title{CAO HW3}
\begin{document}
	\maketitle
	\section*{1}
	The steps for multiplying -7 and 3 are found by closely following the algorithm as stated in \$3.3, p. 187. First 7 is negated using two's complement. One of both of the signs is negative, hence the result must be negative. After using two's complement again on the end result after 4 iterations, the value 1110 1011 (-21) is found.
	\begin{table}[h]
		\centering
		\label{my-label}
		\begin{tabular}{|l|l|l|l|l|}
			\hline
			Iteration & Description & MR  & MD & Product   \\ \hline
			0& Initial values  & 0011  & 1111 1001 &  0000 0000 \\ 
			& 1. Convert to positive using 2s complement & 0011 & 0000 0111 & 0000 0000 \\ \hline
			1& 1. Prod = Prod + MD &  0011  & 0000 0111 & 0000 0111 \\
			& 2. SLL MD &  0011  & 0000 1110 & 0000 0111 \\
			& 3. SRL MR &  0001  & 0000 1110 & 0000 0111 \\ \hline
			
			2 
			& 1. Prod = Prod + MD &  0001  & 0000 1110 & 0001 0101 \\
			& 2. SLL MD &  0001  & 0001 1100 & 0001 0101 \\
			& 3. SRL MR &  0000  & 0001 1100 & 0001 0101 \\ \hline
			
			3 
			& 1. NOP &  0000  & 0001 1100 & 0001 0101 \\
			& 2. SLL MD &  0000  & 0011 1000 & 0001 0101 \\
			& 3. SRL MR &  0000  & 0011 1000 & 0001 0101 \\ \hline

			4 
			& 1. NOP &  0000  & 0001 1100 & 0001 0101 \\
			& 2. SLL MD &  0000  & 0111 0000 & 0001 0101 \\
			& 3. SRL MR &  0000  & 0111 0000 & 0001 0101 \\ \hline
			

			After
			& Convert to negative using 2s complement &  0000  & 0111 0000 & 1110 1011 \\ \hline
		\end{tabular}
	\end{table}
\end{document} 