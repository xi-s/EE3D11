\documentclass[10pt,a4paper]{article}
\usepackage[latin1]{inputenc}
\usepackage{amsmath}
\usepackage{amsfonts}
\usepackage{amssymb}
\usepackage{graphicx}
\usepackage[default,osfigures,scale=0.95]{opensans}
\usepackage[T1]{fontenc}
\author{Remy Jaspers - 4499336}
\title{CAO HW5}
\begin{document}
	\maketitle
	\section*{1.1}
	2-way set associative\\
	4 blocks\\
	1 word / block \\\\
	We have two sets containing two blocks each. Hence we need one bit to select the set. The tag bit is the address size bits(8) - byte offset bits(2) - set bits(1). Therefore, the tag is 5 bits wide.\\\\
	Stripping the tag bits from the address leaves us three bits, one for the index, and the last two for the byte offset.\\\\
	It is unclear whether the set or index bits should be different, because they both mean the same thing, or in other words, one can be derived from the other. I assume the set is being derived from the index \% 2, hence these bits should be equal. \\\\ That the set and index bits are equal is only true in this particular instance, because the number of sets equals the number of blocks in the set. 
\begin{table}[h]
	\centering
	\label{my-label}
	\begin{tabular}{|l|l|l|l|l|l|}
		\hline
		Time & Access  & TAG  & SET  & INDEX  & Byte offset  \\ \hline
		0& 10001101 & 10001 &  1  & 1 & 01\\ \hline
		1& 10110010 & 10110 &  0 & 0 &  10\\ \hline
		2& 10111111 & 10111 &  1 & 1& 11\\ \hline
		3& 10001100 & 10010 &  1 & 1 &00\\ \hline
		4& 10011100 & 10011 &  1 & 1 &00\\ \hline
		5& 11101001 & 11101 &  0 & 0 & 01\\ \hline
		6& 11111110 & 11111 &  1 & 1 & 10\\ \hline
		7& 11101001 & 11101 &  0 & 0 & 01\\ \hline
	\end{tabular}
\end{table}

\section*{1.2}
See table 1
\begin{table}[h]
	\centering
	\caption{Access sequence ex. 1.2}
	\begin{tabular}{|l|l|l|}
		\hline
	Access 0	&  &  \\ \hline
		& Block 0 & Block 1  \\ \hline
	Set 0 	& X & X \\ \hline
	Set 1& 10001 0 0 & XXXXX 1 X \\ \hline
		Access 1&  &  \\ \hline
	& Block 0 & Block 1  \\ \hline
	Set 0	& 10110 0 0 & XXXXX 1 X  \\ \hline
	Set 1&  10001 0 0& XXXXX 1 X  \\ \hline
		Access 2	&  &  \\ \hline
	& Block 0 & Block 1  \\ \hline
	Set 0	& 10110 0 0& XXXXX 1 X  \\ \hline
	Set 1& 10001 1 0 & 10111 0 0  \\ \hline
		Access 3	&  &  \\ \hline
	& Block 0 & Block 1  \\ \hline
	Set 0	& 10110 0 0 & XXXXX 1 X  \\ \hline
	Set 1& 10001 0 1 & 10111 1 0  \\ \hline
		Access 4	&  &  \\ \hline
	& Block 0 & Block 1  \\ \hline
	Set 0	& 10110 0 0  & XXXXX 1 X  \\ \hline
	Set 1& 10001 1 0 & 10011 0 0  \\ \hline
	
		Access 5	&  &  \\ \hline
	& Block 0 & Block 1  \\ \hline
	Set 0	& 10110 1 0  & 11101 0 0  \\ \hline
	Set 1& 10001 1 0& 10011 0 0  \\ \hline
	
		Access 6	&  &  \\ \hline
	& Block 0 & Block 1  \\ \hline
	Set 0	& 10110 1 0  & 11101 0 0  \\ \hline
	Set 1& 11111 0 0 & 10011 1 0  \\ \hline
	
		Access 7	&  &  \\ \hline
	& Block 0 & Block 1  \\ \hline
	Set 0	& 10110 1 0  & 11101 0 1  \\ \hline
	Set 1& 11111 0 0& 10011 1 0  \\ \hline

	\end{tabular}
\end{table}
\section*{1.3}
We have 2 hits out of 8 access, so 2/8 = 25\%
\section*{2}
See tables 2, 3, 4 and 5.
\begin{table}[h]
	\centering
	\caption{LRU}
	\label{my-label}
	\begin{tabular}{|l|l|l|l|l|l|}
		\hline
	Address	& hit/miss  & set 0  & set 0  & set 1  & set 1  \\ \hline
		1& miss  &  &  & 1 & X \\ \hline
		3& miss &  &  & 1 &3  \\ \hline
		5&miss  &  &  & 5 & 3 \\ \hline
		1& miss &  &  & 5 & 1 \\ \hline
		3& miss &  &  & 3 &  1\\ \hline
		1&  hit&  &  & 3 &  1\\ \hline
		3&  hit&  &  & 3 &  1\\ \hline
		5& miss &  &  & 3 & 5 \\ \hline
		3& hit&  &  & 3  & 5 \\ \hline
	\end{tabular}
\end{table}
\begin{table}[h]
	\centering
	\caption{MRU}
	\label{my-label}
	\begin{tabular}{|l|l|l|l|l|l|}
		\hline
		Address	& hit/miss  & set 0  & set 0  & set 1  & set 1  \\ \hline
		1& miss &  &  & 1 & X  \\ \hline
		3& miss &  &  & 1 & 3  \\ \hline
		5& miss &  &  & 1 & 5  \\ \hline
		1& hit &  &  &  1& 5  \\ \hline
		3& miss &  &  &  3&5  \\ \hline
		1&  miss&  &  &  1&5  \\ \hline
		3&  miss&  &  &  3& 5 \\ \hline
		5& hit &  &  &  3&  5 \\ \hline
		3& hit &  &  &  3& 5 \\ \hline
	\end{tabular}
\end{table}
\begin{table}[h]
	\centering
	\caption{random}
	\label{my-label}
	\begin{tabular}{|l|l|l|l|l|l|}
		\hline
		Address	& hit/miss  & set 0  & set 0  & set 1  & set 1  \\ \hline
		1&miss  &  &  &  1 & X  \\ \hline
		3& miss &  &  & 1 & 3  \\ \hline
		5& miss &  &  &  5& 3 \\ \hline
		1& miss &  &  &  5& 1 \\ \hline
		3& miss &  &  &  3& 1 \\ \hline
		1& hit &  &  & 3 &1  \\ \hline
		3&  hit&  &  & 3 & 1 \\ \hline
		5& miss &  &  & 5 & 1  \\ \hline
		3&  miss&  &  &  3& 1 \\ \hline
	\end{tabular}
\end{table}
\begin{table}[h]
	\centering
	\caption{optimal}
	\label{my-label}
	\begin{tabular}{|l|l|l|l|l|l|}
		\hline
		Address	& hit/miss  & set 0  & set 0  & set 1  & set 1  \\ \hline
		1&  miss&  &  &  1 & X  \\ \hline
		3& miss &  &  &  1&3  \\ \hline
		5& miss &  &  &  1&5  \\ \hline
		1& hit &  &  &  1& 5 \\ \hline
		3& miss &  &  &  1& 3 \\ \hline
		1& hit &  &  &  1&3  \\ \hline
		3& hit &  &  &  1&3  \\ \hline
		5&miss  &  &  &  5& 3 \\ \hline
		3& hit &  &  &  5& 3 \\ \hline
	\end{tabular}
\end{table}
\section*{2.5}
This depends too much on how to data is accessed and the size of the data, among other factors. For large datasets, MRU might be more useful if we want to retain older data longer, but then other algorithms might be more feasible depending on the needs of the user.

\section*{3.1}
Offset is 14 bits (page size is $2^{14}$ bits). Virtual address space is $2^{38}$ large dividing these numbers yields 38 - 14 = 24 or $2^{24}$ PTE's. Physical address space is $2^{33}$ byets hence we need 33 - 14 = 19 bits to address the physical memory.
\section*{3.2}
We have $2^{24}$ entries which are each $2^{2}$ bytes wide, hence $2^{26}$ = 67 megabytes (or 64 mibibytes)
\section*{3.3}
32 bits from the PTE, take off the 10 reserved bits leaves 22 bits. The page offset is 14 bits, so in total we have 22+14 = 36 or $2^{36}$ = 68 gigabytes (64 gibibytes).
\end{document} 